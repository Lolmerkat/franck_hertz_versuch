\documentclass[a4paper,11pt]{article}

\usepackage[utf8]{inputenc}
\usepackage[T1]{fontenc}
\usepackage[ngerman]{babel}
\usepackage{amsmath}
\usepackage{lmodern}
\usepackage{hyperref}
\usepackage{geometry}
    \geometry{
    top = 1cm,
    left= 2.5cm,
    }

\title{Franck-Hertz-Versuch\\ \large (und Absorption \& Emission)}
\author{Jan M.}
\date{}

\begin{document}


\maketitle
\section{Zweck}
Der Franck-Hertz-Versuch wurde durchgeführt, um die \textit{Existenz von diskreten Energieniveaus in Atomen} nachzuweisen.

\section{Warum?}
Durch den Franck-Hertz-Versuch fand man heraus, dass nur bei bestimmten kinetischen Energiemengen der Zusammenstoß von Elektronen mit Hg-Atomen als inelastischer Stoß stattfindet. Das heißt, nur bei bestimmten Energien wird beim Zusammenstoß Energie übertragen.

\vspace{0.5em}
Man spricht hier vom sogenannten \textit{Anregen}, da Elektronen in ein höheres Energieniveau versetzt werden. Bei Quecksilberatomen betragen die Energieniveausabstände ca. 4,9\,eV. Dabei können die Elektronen ihre \textit{komplette} Energie abgeben, niemals nur Teile davon.

\section{Emission \& Absorption}

\subsection{Emission}
Nach einer unbestimmten Zeit fallen die angeregten Elektronen wieder in ihr Grundniveau zurück und geben dabei Licht mit einer bestimmten Wellenlänge ab. Dieser Vorgang wird als \textbf{Emission} bezeichnet.\\[1ex]
Mithilfe der folgenden Herleitung lässt sich die Wellenlänge des emittierten Lichts aus der Photonenenergie bestimmen:
\begin{align*}
E_{ph} &= E_{2} - E_{1} \quad \text{mit} \quad E_{ph} = h \cdot f,\\[1ex]
h \cdot f &= E_{2} - E_{1},\\[1ex]
\Leftrightarrow\quad f &= \frac{E_{2} - E_{1}}{h},\\[1ex]
\intertext{eingesetzt in die Formel $\lambda = \frac{c}{f}$ ergibt sich:}
\lambda &= \frac{c \cdot h}{E_{2} - E_{1}}.
\end{align*}

\subsection{Absorption}
Das gleiche Prinzip der Anregung kann auch durch Photonen erfolgen. Bei diesem Vorgang werden Atome durch einfallendes Licht angeregt, wobei das Licht \textbf{absorbiert} wird. Dabei übertragen die Photonen ihre \textbf{gesamte} Energie. Die angeregten Elektronen fallen anschließend wieder in ihr Grundniveau zurück und emittieren Licht mit der aufgenommenen Energie. Da diese Emission unabhängig von der Einfallsrichtung erfolgt, resultiert dies in einem charakteristischen Muster der absorbierten Wellenlängen.

\end{document}

